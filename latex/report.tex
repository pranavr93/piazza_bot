\documentclass[sigconf]{acmart}

\usepackage{booktabs} % For formal tables


% Copyright
%\setcopyright{none}
%\setcopyright{acmcopyright}
%\setcopyright{acmlicensed}
\setcopyright{rightsretained}
%\setcopyright{usgov}
%\setcopyright{usgovmixed}
%\setcopyright{cagov}
%\setcopyright{cagovmixed}


% DOI
\acmDOI{10.475/123_4}

% ISBN
\acmISBN{123-4567-24-567/08/06}

%Conference
\acmConference[WOODSTOCK'97]{ACM Woodstock conference}{July 1997}{El
  Paso, Texas USA} 
\acmYear{1997}
\copyrightyear{2016}

\acmPrice{15.00}


\begin{document}
\title{Jarvis - A Smart Bot for Piazza}
\author{Pranav Ramarao, Isaac Bowen, Yu Ke}
\orcid{1234-5678-9012}
\affiliation{%
  \institution{University of Michigan}
  \city{Ann Arbor} 
}
\email{{pranavr, irbowen, yuke}@umich.edu}


% The default list of authors is too long for headers}
\renewcommand{\shortauthors}{B. Trovato et al.}


\begin{abstract}
In this work, we explore some of the challenges faced in Piazza specially for large sized classes. We present a smart bot - Jarvis that can tackle the problems by using state-of-the-art IR techniques. We show the Jarvis can greatly increase the efficiency of the instructors and also serve as a useful aid for students.
\end{abstract}

\keywords{IR, piazza, search, similarity,FAQ generation}

\maketitle

\section{Problem}
Piazza is a great tool for students and instructors to interact with teach other. It allows students to post questions on a classwide forum, where other students and instructors can answer the questions. For small classes, where the question volume is very low, works well. Instructors can respond quickly, and scale isn't an issue. However, as the departments grow and class sizes increase, it becomes difficult to answer all the questions quickly. For our project, we wanted to focus on a few specific issues related to Piazza at scale.
\subsection{Duplicate Questions}
As one of the side effects of the number of students posting questions, it becomes difficult for other students in the class to know if their question as already been asked. Because of the question volume, they may not have time to read every question that is asked, and as a result, repost a question that is very similar to one that has already been asked.

\subsection{Top Questions}
A related issue is that students aren't getting the most they could out of Piazza because of the number of questions. Students can't dedicate hours every day to read through all questions posted every day. Also, many of the question are very student-specific (they won't related to other students, so the whole class doesn't benifit from reading them). However, there are some questions that are very common, or exceptionally helpful, and it would be beneficial to aggregate these high quality questions.

\subsection{Better Search Features}
Todo


\bibliographystyle{ACM-Reference-Format}
\bibliography{sigproc} 

\end{document}
