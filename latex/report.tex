\documentclass[sigconf]{acmart}

\usepackage{booktabs} % For formal tables


% Copyright
%\setcopyright{none}
%\setcopyright{acmcopyright}
%\setcopyright{acmlicensed}
\setcopyright{rightsretained}
%\setcopyright{usgov}
%\setcopyright{usgovmixed}
%\setcopyright{cagov}
%\setcopyright{cagovmixed}


% DOI
\acmDOI{10.475/123_4}

% ISBN
\acmISBN{123-4567-24-567/08/06}

%Conference
\acmConference[WOODSTOCK'97]{ACM Woodstock conference}{July 1997}{El
  Paso, Texas USA} 
\acmYear{1997}
\copyrightyear{2016}

\acmPrice{15.00}


\begin{document}
\title{Jarvis - A Smart Bot for Piazza}
\author{Pranav Ramarao, Isaac Bowen, Yu Ke}
\orcid{1234-5678-9012}
\affiliation{%
  \institution{University of Michigan}
  \city{Ann Arbor} 
}
\email{{pranavr, irbowen, yuke}@umich.edu}


% The default list of authors is too long for headers}
\renewcommand{\shortauthors}{B. Trovato et al.}


\begin{abstract}
In this work, we explore some of the challenges faced in Piazza specially for large sized classes. We present a smart bot - Jarvis that can tackle the problems by using state-of-the-art IR techniques. We show the Jarvis can greatly increase the efficiency of the instructors and also serve as a useful aid for students.
\end{abstract}

\keywords{IR, piazza, search, similarity,FAQ generation}

\maketitle

\section{Problem}
Piazza is a great tool for students and instructors to interact with teach other. It allows students to post questions on a class wide forum, where other students and instructors can answer the questions. For small classes, where the question volume is very low, works well. Instructors can respond quickly, and scale isn't an issue. However, as the departments grow and class sizes increase, it becomes difficult to answer all the questions quickly. For our project, we wanted to focus on a few specific issues related to Piazza at scale. The authors our instructors for undergraduate courses at UofM - these course have roughly 400 and 600 students, and only and handful of staff members to manage Piazza. We are interested in solving these problems to assists in our courses.

\subsection{Duplicate Questions}
As one of the side effects of the number of students posting questions, it becomes difficult for other students in the class to know if their question as already been asked. Because of the question volume, they may not have time to read every question that is asked, and as a result, repost a question that is very similar to one that has already been asked. For instructors, this means that they have to answer the same question over and over, or link students to the previous question. For students, it could provide confusion, as many very similar questions have very similar answers, but students are quite sure if its the same question (and therefore has the same answer).

\subsection{Top Questions}
A related issue is that students aren't getting the most they could out of Piazza because of the number of questions. Students can't dedicate hours every day to read through all questions posted every day. Also, many of the question are very student-specific (they won't related to other students, so the whole class doesn't benifit from reading them). However, there are some questions that are very common, or exceptionally helpful, and it would be beneficial to aggregate these high quality questions.

\subsection{Better Search Features}
Piazza's search features leave something to be desired. The largest problem seems to be that the results are very heavily skewed towards the most recent results, when that may not actually be what the student is looking for. We wanted to provide a method for students to more accurately search Piazza for their questions.

\section{Introduction}

Piazza is a free online gathering place where students can ask and answer questions 24/7 under the guidance of their instructors. Collaboration occurs in real time, which encourages participation and fuels discussion. Leading campuses across North America have a significant Piazza presence. Larger courses typically have over 500 students getting involved in the feed. Some of the painpoints however are :
\begin{enumerate}
\item students tend to ask duplicate questions without searching for them first.
\item students are overwhelmed by the huge number of questions on piazza specially in larger courses.
\item the search facilities that piazza provides are average and rely on recency based features and fail to consider relevance based ones.
\end{enumerate}

\subsection{Piazza}
\subsection{Chat Bots}

\section{Method//Approach}
We designed and implmented Jarvis, a bot written in python that can accomplish these goals. 

\subsection{System Overview}
Jarvis works by indexing all of the post on a given course on Piazza. On startup, it will bulk download all of the questions posted previously and constructed an index from there. After that, it sits in a loop, waiting for new questions to be posted. 

\subsection{Tools and packages used}
We took advantage of several python packages and libaries to make the devolpment of Jarvis faster and the results more correct. We used python whoosh for indexing the documents. This allowed us to index the documents, but first we actually had to collect them in one place. To do this, we needed to download all olf the questions off of Piazza. Piazza does not have an offical supported API, but there is an unoffical API that has been constructed through some reverse engineering. While this did not work as perfectly as a fully supported first party API would have, the results were surprisingly good and required only a small amount of work on our part to work with.

\subsection{Indexing and Feature Extraction}

\subsection{Data Preprocessing}

\subsection{Duplicate Detection}

\subsection{Top Questions}
\subsection{Improved Search Feature}


\bibliographystyle{ACM-Reference-Format}
\bibliography{sigproc} 

\end{document}
